\mode<article>{\usepackage{fullpage}}
\mode<presentation>{\usetheme{shurain}}


\usepackage{kotex}
\usepackage{multirow}
\usepackage{tikz}
\usepackage{graphicx}
\usepackage[normal, tight, center]{subfigure}
\usepackage{amssymb}
\usepackage{pgf}
\usepackage{url}
\usepackage{hyperref}
\hypersetup{colorlinks=true}
\usepackage{mathrsfs}
\usepackage{mathtools}
\usepackage{algorithm2e}
\usepackage{setspace}


\DeclareMathOperator*{\argmin}{\arg\!\min}
\DeclareMathOperator*{\argmax}{\arg\!\max}

\include{pygments}

\setbeamercovered{transparent}

\def\supertiny{ \font\supertinyfont = cmr10 at 7pt \relax \supertinyfont}


\title[\hspace{2em}\insertframenumber/\inserttotalframenumber]{\huge{Title}}
\subtitle{Subtitle}
\author[Sungjoo Ha]{Sungjoo Ha}
\date{July 10th, 2016}

% \institute{
%     Optimization Lab \\
%     Seoul National University \\
%     Computer Science \\
%     GECCO 2015
% }

\begin{document}

\frame{
\maketitle
}


\begin{frame}
    \frametitle{Title}
\end{frame}


\begin{frame}[fragile]
    \frametitle{Zen of NumPy\footnote[frame]{\url{https://github.com/numpy/numpy/issues/2389}}}
{\scriptsize
등간격이 흩어진 것보다 낫다. \\
\qquad \qquad {\color{gray}Strided is better than scattered} \\
연속된 것이 등간격보다 낫다. \\
\qquad \qquad {\color{gray}Contiguous is better than strided} \\
원하는 것을 설명하는 편이 명령을 내리는 것보다 낫다 (데이터 타입을 사용하자). \\
\qquad \qquad {\color{gray}Descriptive is better than imperative (use data-types)} \\
배열 지향이 대체로 객체 지향보다 낫다. \\
\qquad \qquad {\color{gray}Array-oriented is often better than object-oriented} \\
브로드캐스팅은 좋은 아이디어다 -- 가능하면 사용하자. \\
\qquad \qquad {\color{gray}Broadcasting is a great idea -- use where possible} \\
벡터화한 것이 명시적인 루프보다 좋다. \\
\qquad \qquad {\color{gray}Vectorized is better than an explicit loop} \\
하지만 복잡하다면 numexpr, weave나 Cython을 사용해라. \\
\qquad \qquad {\color{gray}Unless it's complicated --- then use numexpr, weave, or Cython} \\
고차원에서 생각하라. \\
\qquad \qquad {\color{gray}Think in higher dimensions} \\
}
\end{frame}


\end{document}
